\documentclass[12pt]{article}

\usepackage[magyar]{babel}
\usepackage{t1enc}
\usepackage{times}
% \usepackage{lipsum}
\usepackage{graphicx}
\usepackage[]{hyperref}
\hypersetup{
    colorlinks=true,
    linkcolor=blue,
    filecolor=magenta,
    urlcolor=cyan
}
\usepackage[backend=biber]{biblatex}
% \usepackage[]{biblatex}
\addbibresource{hivatkozasok.bib}

\usepackage{listings}

\usepackage{xcolor}

\definecolor{codegreen}{rgb}{0,0.6,0}
\definecolor{codegray}{rgb}{0.5,0.5,0.5}
\definecolor{codepurple}{rgb}{0.58,0,0.82}
\definecolor{backcolour}{rgb}{0.95,0.95,0.92}

\lstdefinestyle{mystyle}{
    backgroundcolor=\color{backcolour},
    commentstyle=\color{codegreen},
    keywordstyle=\color{magenta},
    numberstyle=\tiny\color{codegray},
    stringstyle=\color{codepurple},
    basicstyle=\ttfamily\footnotesize,
    breakatwhitespace=false,
    breaklines=true,
    captionpos=b,
    keepspaces=true,
    numbers=left,
    numbersep=5pt,
    showspaces=false,
    showstringspaces=false,
    showtabs=false,
    tabsize=2
}

\lstset{style=mystyle}

%A fejléc láblécek kialakításához:
\usepackage{fancyhdr}

%Margók:
\hoffset -1in
\voffset -1in
\oddsidemargin 35mm
\textwidth 150mm
\topmargin 15mm
\headheight 10mm
\headsep 5mm
\textheight 237mm

\author{Südi Tamás}
\title{Terhelés-kiegyenlítés AP-asszisztált roaming segítségével OpenWrt-en}
\makeatletter

\begin{document}

%A FEJEZETEK KEZDŐOLDALAINAK FEJ ES LÁBLÉCE:
%a plain oldalstílust kell átdefiniálni, hogy ott ne legyen fejléc:
\fancypagestyle{plain}{%
%ez mindent töröl:
\fancyhf{}
% a láblécbe jobboldalra kerüljön az oldalszám:
\fancyfoot[R]{\thepage}
%elválasztó vonal sem kell:
\renewcommand{\headrulewidth}{0pt}
}

%A TÖBBI OLDAL FEJ ÉS LÁBLÉCE:
\pagestyle{fancy}
\fancyhf{}
\fancyhead[L]{\@title}
\fancyfoot[R]{\thepage}


%A címoldalra se fej- se lábléc nem kell:
\thispagestyle{empty}

\begin{center}
\vspace*{1cm}
{\Large\bf Szegedi Tudományegyetem}

\vspace{0.5cm}

{\Large\bf Informatikai Intézet}

\vspace*{3.8cm}


{\LARGE\bf \@title}


\vspace*{3.6cm}

{\Large Szakdolgozat}

\vspace*{4cm}

%TODO: supervisorok hozzáadása
{\large
\begin{tabular}{c@{\hspace{4cm}}c}
	\emph{Készítette:}     &\emph{Témavezető:}\\
	\bf{\@author}  &\bf{??????.?}\\
	programtervező informatikus szakos     &egyetemi docens\\
	hallgató&
\end{tabular}
}

\vspace*{2.3cm}

{\Large
Szeged
\\
\vspace{2mm}
\the\year{}
}
\end{center}


%A tartalomjegyzék:
\tableofcontents

\section*{Feladatkiírás}


A témavezető által megfogalmazott feladatkiírás. Önálló oldalon szerepel.
\newpage

\section*{Tartalmi összefoglaló}

% A tartalmi összefoglalónak tartalmaznia kell (rövid, legfeljebb egy oldalas, összefüggő megfogalmazásban)
% a következőket: a téma megnevezése, a megadott feladat megfogalmazása - a feladatkiíráshoz viszonyítva-,
%a megoldási mód, az alkalmazott eszközök, módszerek, az elért eredmények, kulcsszavak (4-6 darab).
%
%Az összefoglaló nyelvének meg kell egyeznie a dolgozat nyelvével. Ha a dolgozat idegen nyelven készül,
%magyar nyelvű tartalmi összefoglaló készítése is kötelező (külön lapon), melynek terjedelmét a TVSZ szabályozza.

A dolgozat célja az AP-asszisztált roaming megvalósítása OpenWrt alapú rendszerekre.
A dolgozat célja továbbra, hogy bemutassa az OpenWrt rendszert, a roaming technikákat és a szükséges 802.11 protokollokat.

A dolgozat első fejezete az OpenWrt rendszer bemutatásával és telepítésével foglalkozik. Összehasonlítja a virtuális fejlesztői környezet előnyeit és hátrányait a fizikai környezettel, telepítés, használat, sebesség és konfigurálhatóság szempontjából.
A dolgozat második fejezete a hálózati topológiákkal, a roaming technikákkal és a szükséges 802.11 protokollok bemutatásával foglalkozik, ismerteti az AP-asszisztált roamingot.
A dolgozat harmadik fejezete a terhelés-kiegyenlítési algoritmusokkal foglalkozik.
A dolgozat negyedik részében az éles környezetben történő tesztelésről és a kapott eredményekről szól.
A dolgozat utolsó fejezete a szakirodalmat és a kapcsolódó projekteket mutatja be.

\newpage

\section{OpenWrt}

Az OpenWrt egy Linux-alapú, nyílt forráskódú, hálózati eszközökhöz készült operációs rendszer.

Az OpenWrt a gyártói firmware helyett telepíthető a támogatott eszközökön. Ahhoz képest számos előnyt kínál a felhasználóknak, beleértve a letisztultságot, a nagyobb testreszabhatóságot, több funckiót és jobb biztonságot.

A legtöbb komponens és a build rendszer a GNU General Public License Version 2 licensz alatt érhető el, azonban néhány, elsősorban a nem OpenWrt-ben létrehozott részek más licensek alatt állnak. \cite{openwrt_about} \cite{openwrt_faq} \cite{openwrt_home}

\subsection{Telepítés fizikai eszközre}

Egyes eszközök már rendelkeznek OpenWrt vagy OpenWrt alapú firmwarrel, azonban legtöbbször ennek telepítése a felhasználó feladata.

A telepítés folyamata eszközönként eltérő lehet, de általában az alábbi módokon történhet: az eszköz webes kezelőfelületen keresztül, FTP-n keresztül, SD-kártya vagy USB-meghajtó segítségével, soros port használatával.

A támogatott eszközök listája a \url{https://openwrt.org/toh/start} oldalon található. Itt az eszköz támogatottságától és népszerűségétől függően megtalálhatóak annak specifikációi, a hozzá tartozó firmwarek letöltési linkje és a telepítési, visszaállítási útmútatók.


\cite{openwrt_install}
\cite{openwrt_stock}

\subsection{Telepítés virtuális környezetbe}

A virtuális környezet egy olyan szoftveres megoldás, ami lehetővé teszi azt, hogy a felhasználó egyszerre futtasson több, akár különböző operációs rendszert is a számítógépén.

A virtuális környezet rengeteg előnnyel járhat egy fejlesztő számára, mint például az extra eszköz használatának elkerülése, az egyszerűbb fájlátvitel, a kijelző és a billentyűzet használata, a gyorsabb hardver, mentések készítése és visszaállítása és ezek megosztása más fejlesztőkkel. Azonban az ilyen környezeteknek is lehetnek hátrányai, mint például, hogy a hálózati kártya nem rendelkezik a szükséges hardveres támogatással és nem olyan megbízható a teljesítménye, mint egy erre tervezett eszköznek.

\subsubsection{Virtuális környezet Linux alatt}

A Linux-alapú operációs rendszerek népszerűek a fejlesztők körében, mivel ingyenesek és számos olyan funkcióval rendelkeznek, amelyek lehetővé teszik a hatékony és kényelmes munkavégzést.

Ezen a platformon több virtuális környezet is elérhető, mint például a VirtualBox, a VMware Workstation vagy a KVM.

A nyílt forráskódú Kernel-based Virtual Machine (KVM) egyik előnye, hogy képes
átadni PCI csatlakozású eszközöket is virtuális gépnek, így alacsony szintű hozzáférést biztosít a hardverhez. \cite{kvm} \cite{pci_passthrough}

Több virtuális környezet kezelő is támogatja mind az USB, mind a PCI csatlakozású eszközök átadását. Én ezek közül a virt-manager nevű programot választottam.


Hogy egyszerűsítsem a telepítési folyamatot \href{https://github.com/uwzis/gpu-passthrough-manager}{gpu-passthrough-manager} nevű szoftvert használtam, amely segítségével felkészítettem a rendszeremet a PCI csatornán keresztül csatlakoztatott WLAN-vezérlő átadására.

Ezután már csak meg kellett adnom a grub rendszerbetöltőnek, hogy a rendszert a vfio-pci.ids=8086:06f paraméterrel indítsa el, amely a WLAN-vezérlő azonoítóját jelöli.

% Ehhez ugye nem kell forrást megjelölnöm?

A számítógépem specifikációjának megfelelően a \href{https://downloads.openwrt.org/releases/22.03.2/targets/x86/64/openwrt-22.03.2-x86-64-generic-ext4-combined.img.gz}{openwrt-22.03.2-x86-64-generic-ext4-combined.img} képet használtam a virtuális gép létrehozásához. Ez nem tartalmaz semmilyen telepítőt, helyette grub rendszerbetöltő segítségével indítja el a rendszert.

\subsubsection{Virtuális környezet Windows alatt WSL2 segítségével}

A Windows Subsystem for Linux (WSL) egy olyan alkalmazás, amely lehetővé teszi a Linux-alapú operációs rendszerek futtatását Windows alatt. A WSL2 egy újabb verziója a WSL-nek, amely a Linux kernelt futtatja a Windows alatt, így alacsonyabb szintű hozzáférést biztosít a hardverhez. \cite{wsl2}

Bár a WSL2 a Hyper-V alapú virtualizáció segítségével egy tényleges Linux rendszert futtat, mégsem képes sem a PCI csatornán keresztül csatlakoztatott eszközöket, sem az USB csatlakozású eszközöket átadni a virtuális gépnek. Bár elméletileg lehetséges lenne a külső eszközök átadása a virtuális rendszernek USB over IP protokoll használatával, de a gyakorlatban ez nehezen elvégezhető, mivel ennek a protokollnak kifejlesztésekor a cél eszközök nem hálózati kontrollerek, hanem perifériák és tárolóeszközök voltak. \cite{wsl2_usbip}

\subsubsection{Virtuális környezet Windows alatt Oracle VirtualBox segítségével}

Az Oracle VirtualBox egy alternatív virtális környezet, amely használatához először ki le kell tiltani a WSL környezetet, mivel a kettő nem egyidejűleg futtatható. Ezután telepíthető a VirtualBox.

Ezt legegyszerűbben a host rendszeren futó parancsorból lehet elvégezni:

\begin{lstlisting}[language=Bash]
    bcdedit /set hypervisorlaunchtype off
\end{lstlisting}


Ez a parancs letiltja a WSL indítását, így az Oracle VirtualBox is futtatható lesz a gépen. A módosítások érvényesítéséhez újra kell indítani a gépet. \cite*{virtualbox_and_wsl}

Ámbár az Oracle Virtualbox nem támogatja a .img kiterjesztésű képeket, de tartalmazza a VBoxManage nevű programot, amely segítségével az img kép átkonvertálható .vdi kiterjesztésű virtuális lemezzé.

\begin{lstlisting}[language=Bash]
    & 'C:\Program Files\Oracle\VirtualBox\VBoxManage.exe' convertfromraw  --format VDI '.\openwrt-22.03.3-x86-64-generic-ext4-combined.img' '.\openwrt.vdi'
\end{lstlisting}

Az újonnan létrejött openwrt.vdi kép segítségével létrehozható a virtuális gép. Alapértelmezetten a host gépről nem érhető el a virtuális gép hálózata, azonban ez megoldható port-forwarding szabályok felvétele segítségével a Virtuaális gép beállításainak Hálózat / adapter1 / speciális / port forwarding menüpontjában.

\subsection{Hálózat beállítása OpenWrt rendszeren}

\subsection{A WAN port beállítása}

\subsection{A WLAN beállítása}

%laptörés:
\newpage

\section*{Fuggelek}
\subsection{A program forráskódja}
A függelékbe kerülhetnek a hosszú táblázatok, vagy mondjuk egy programlista:
% \begin{lstlisting}[language=Python]
% for i in range(3):
% 	print(i)
% \end{lstlisting}


\section*{Nyilatkozat}
\noindent
Alulírott \@author, programtervező informatikus szakos hallgató, kijelentem, hogy a dolgozatomat a Szegedi Tudományegyetem, Informatikai Intézet XY Tanszékén készítettem, XY  diploma megszerzése érdekében.
%TODO tanszek es diplomahoz valami
Kijelentem, hogy a dolgozatot más szakon korábban nem védtem meg, saját munkám eredménye, és csak a hivatkozott forrásokat (szakirodalom, eszközök, stb.) használtam fel.

Tudomásul veszem, hogy szakdolgozatomat a Szegedi Tudományegyetem Diplomamunka Repozitóriumában tárolja.

\vspace*{2cm}

\begin{tabular}{lc}
Szeged, \today\
\hspace{2cm} & \makebox[6cm]{\dotfill} \\
& aláírás \\
\end{tabular}

\section*{Köszönetnyilvánítás}

Ezúton szeretnék köszönetet mondani \textbf{X. Y-nak} ezért és ezért \ldots

%\section*{hivatkozások}


\printbibliography

%\bibliography{hivatkozasok}{}
% \bibliographystyle{plain}
% \addcontentsline{toc}{section}{Hivatkozások}

\end{document}
